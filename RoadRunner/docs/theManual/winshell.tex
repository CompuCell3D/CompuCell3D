\chapter{WinShell}
\label{chap2}
\thispagestyle{empty}

\section{What is WinShell ?}
\label{whatisws}

WinShell is a graphical user interface for easy working
with \LaTeX\enspace or \TeX. It includes a text editor, different
tool bars and user configuration options.
It is NOT a \LaTeX\enspace system.
An additional \LaTeX\enspace package for Windows is needed.

\subsection{Features}
\label{features}

%% todo: more features

Some of the features are:
Multi language support
(Brazilian-Portuguese, Catalan, Chinese, Czech, Danish, Dutch, English, French,
Galician, German, Hungarian, Italian, Japanese, Mexican-Spanish, Polish, Portuguese,
Russian, Spanish, Swedish and Turkish);
Project environment (Table of Contents, Figures, Tables, Bibliography);
Built-in spell checker;
Bibliography support;
Forward and inverse search;
Table wizard;
One-instance-program;
Multiple documents;
Project and Output Window;
User defined programs;
Macros;
Configure toolbars (symbols, user-def. programs, macros);
Choose font;
Windows/Unix file format;
WinShell starts command line driven;
Syntax highlighting;
Wrap mode;
Unicode support;
Drag \& drop;
Portable WinShell.
